% Template for Cogsci submission with R Markdown

% Stuff changed from original Markdown PLOS Template
\documentclass[10pt, letterpaper]{article}

\usepackage{cogsci}
\usepackage{pslatex}
\usepackage{float}
\usepackage{caption}

% amsmath package, useful for mathematical formulas
\usepackage{amsmath}

% amssymb package, useful for mathematical symbols
\usepackage{amssymb}

% hyperref package, useful for hyperlinks
\usepackage{hyperref}

% graphicx package, useful for including eps and pdf graphics
% include graphics with the command \includegraphics
\usepackage{graphicx}

% Sweave(-like)
\usepackage{fancyvrb}
\DefineVerbatimEnvironment{Sinput}{Verbatim}{fontshape=sl}
\DefineVerbatimEnvironment{Soutput}{Verbatim}{}
\DefineVerbatimEnvironment{Scode}{Verbatim}{fontshape=sl}
\newenvironment{Schunk}{}{}
\DefineVerbatimEnvironment{Code}{Verbatim}{}
\DefineVerbatimEnvironment{CodeInput}{Verbatim}{fontshape=sl}
\DefineVerbatimEnvironment{CodeOutput}{Verbatim}{}
\newenvironment{CodeChunk}{}{}

% cite package, to clean up citations in the main text. Do not remove.
\usepackage{apacite}

% KM added 1/4/18 to allow control of blind submission


\usepackage{color}

% Use doublespacing - comment out for single spacing
%\usepackage{setspace}
%\doublespacing


% % Text layout
% \topmargin 0.0cm
% \oddsidemargin 0.5cm
% \evensidemargin 0.5cm
% \textwidth 16cm
% \textheight 21cm

\title{Caregiver reconstruction of children's errors: the preservation of
complexity in language (improve title)}


\author{Madeline Meyers \and Daniel Yurovsky \\
        \texttt{\{mcmeyers, yurovsky\}@uchicago.edu} \\
       Department of Psychology \\ University of Chicago}

\begin{document}

\maketitle

\begin{abstract}
Why do languages change? One possibility is they evolve in response to
two competing pressures: (1) to be easily learned, and (2) to be
effective for communcation. In a number of domains (e.g.~kinship
categories, color terms), variation in the world's natural languages
appears to be accounted for by different but near-optimal tradeoffs
between these two pressures (Regier, Kemp, \& Kay, 2015). Models of tis
evolutionary process have focused on transmission chains in which errors
of learning by one agent become the language input for the next
generation. However, a critical feature of human language is that
children do not learned in isolation, but rather in communicative
interactions with caregivers who can draw inferences from a child's
errorful productions to their intended interests. In a set of iterated
learning experiments, we show that this supportive context can have a
powerful stabilizing role in the development of artificial languages,
allowing them achieve higher levels of assymptotic complexity than they
would by transmission alone.

\textbf{Keywords:}
communication; language acquisition; language evolution; iterated
learning
\end{abstract}

\section{Introduction}\label{introduction}

How do you ask a group of people where they are going in Spanish? In
Spain, the answer depends on the group: you might ask ``Donde van
ustedes?'' of a group of work colleagues, but to address your friends,
you use the informal ``Donde váis vosotros?'' instead. In Mexican
Spanish, this distinction has disappeared, and the ``ustedes'' form is
used exclusively. Why did Spanish change in this way, simplifying and
shedding the formal second person plural? One working theory is that
languages evolve to adapt to two dynamic competing pressures: (1) ease
of learning and transmission, and (2) effective communication (Lupyan \&
Dale, 2010).

When children are learning language, they often make simplification
errors (Bowerman, 1982). If a child is asking for some milk, they may
ask for a ``baba'' (bottle). Thus, the child has coined a new word,
which her parents understand. This simplification works well for the
child, until she starts to learn her animal words, and calls a sheep
``baba''. The child language-learner has shown the effects of the
simplicity pressure in language: if she calls both bottles and sheep
``baba'', she has to learn one less word. Does she go through her life,
believing that ``baba'' is the label for these disparate objects? It's
possible, if she never has a need to discriminate between them, she
might even pass this label on to her children. In this way, errors in
language can be passed on to the next generation, through transmission
from one speaker to the next. It reflects the needs of the language
speakers: if our hypothetical language learner grew up in a world where
there was no need to distinguish between bottles and sheep, there is no
need to retain this extra complexity (and extra memory load).

Children are often the actors who drive language evolution (Senghas,
2003), yet they differ from adults in their cognitive capabilities,
namely, memory systems (Kempe, Gauvrit, \& Forsyth, 2015), interests and
early vocabularies, and conversation partners. Therefore, though
children are skilled language learners, their developing cognitive
systems prevent aspects of language that are difficult to learn and
remember from being passed on---pushing languages towards simplicity
(Hudson Kam \& Newport, 2005; Senghas, 2003). But, languages that become
too simple can lose the ability to be effective for communication
(Kirby, Griffiths, \& Smith, 2014). What enables languages to retain
their communicative utility in the face of these learnability pressures?

Indeed, most Americans grow up in a world where it is useful to
discriminate sheep and bottles -- and they must learn the different
labels for these objects. It is often caregivers, through their explicit
interventions as well as their implicit modeling of correct language,
who may be reintroducing descriptiveness into a language where it would
otherwise be lost. Children's language learning is greatly influenced by
those around them -- especially the adults they talk to most. These
caregivers control much of their child's linguistic input, and are
responsible for seeing that their children develop effective and useful
systems of communication. Even the youngest children are not passive
learners of language --- they are active participants, engaging in
conversations with their parents. These adults are experts both in the
language and in the children themselves, as they understand the child's
intuitions, personality, and context. Caregivers play an important
interpretive role in these interactions through their ability to
understand the intended target of children's errorful productions
(Chouinard \& Clark, 2003). Adults can explictly correct their
children's language errors in various ways (e.g., by interruptions or
repeating the correct word/grammatical form) (Penner, 1987). Yet,
children primarily learn language through listening to others talk,
rather than explicit instruction (Romberg \& Saffran, 2010). Thus,
parent's modeling of accurate language constructions can have a powerful
effect on reducing children's language errors: over time, children fix
their own mistakes because they have learned the correct constructions
from their caregivers (Hudson Kam \& Newport, 2005). By way of this
feedback, both implicit and explicit, children's simplification errors
are corrected, and children are able to acquire adult-like speech.
Eventually, when a child grows to be an adult, they will not transmit
the errors they had as a child, but the corrct forms of speech they
learned from their caregivers -- as long as learning the correct forms
is useful and necessary. Thus, over the course of a lifetime, the child
language learner grows to become a parent language teacher, correcting
their own children's errors. These error reconstructions may be a
mechanism by which more structure is retained in language over many
lifetimes than children could sustain alone.

\subsection{Using iterated learning to study language
change}\label{using-iterated-learning-to-study-language-change}

In order to study the effects of the descriptiveness and
transmissibility biases on inter-generational language evolution, we
will be using the iterated learning paradigm (Kirby et al., 2014). In an
iterated learning paradigm, one participant is trained on a
randomly-generated language---for example, a set of words created by
arbitrarily pairing syllables together. The participant is later asked
to recall the language, and their responses are given as training input
for the next subject, thus creating a transmission chain. This iterated
process mimics the transmission of language across generations, with
each participant unintentionally changing the language through their
memory biases. The vast majority of iterated learning studies have used
adult participants (Christiansen \& Kirby, 2003; Kirby, Dowman, \&
Griffiths, 2007; Kirby et al., 2014; e.g., Smith \& Wonnacott, 2010;
structure \& signals, 2014), while few studies have used children as
research subjects (Kempe et al., 2015; Raviv \& Arnon, 2018). However,
language evolution cannot be fully grasped using this paradigm with only
separate adult or child learning chains, because language learning does
not occur only within the same age group (horizontal transmission), or
only across age groups (vertical transmission), but it occurs
dynamically, in both directions. In a true language-acquisition
environment, a child receives both language input and feedback from
their caregiver and uses it to interact with their peers throughout
life, eventually growing into a new teacher-caregiver.

We adapted Kempe et al. (2015)'s child-friendly language paradigm to
model the effect of a cooperative caregiver on the evolution of
language. We hypothesize that these error-correctors (analogus to
caregivers and teachers) are pivotal not only to an individual's
successful language acquisition, but also to the evoulution of a
language as a whole. This is because those who correct mistakes and
provide feedback are able to protect against the transmissibility
(simplicity) bias, which is likely stronger in early language learners,
by re-introuding and preserving complexity in language.

All experiments were pre-registered on Open Science Framework, and all
data and code can be accessed at the following link:
\url{https://osf.io/guzyf/}.

\section{Experiment 1: Replicating Kempe et al.
(2015)}\label{experiment-1-replicating-kempe-2015}

In a baseline experiment, adults participated in a standard baseline
iterated learning study, using stimuli adapted from Kempe et al. (2015).
Participants were told to reproduce patterns on grids, and each user's
responses were used as training input for the subsequent participant.

\subsection{Method}\label{method}

\subsubsection{Participants}\label{participants}

Participants in Experiment 1 were 120 adults recruited on Amazon
Mechanical Turk. These participants were dived into twenty diffusion
chains, each of which had six generations. Each participant gave
informed consent. The task was approximately eight minutes long, and
subjects were compensated \$0.50 for their participation.

\subsubsection{Design and Procedure}\label{design-and-procedure}

\textbf{I THINK YOU MIGHT WANT TO DESCRIBE GENERAL TRIALS INSTEAD,
ACTUALLY. AND THEN SAY WE STARTED WITH 3 PRACTICE TRIALS} Participants
in Experiment 1 were asked to re-create patterns on a grid. Participants
provided consent. After a consent screen, participants first viewed a
training trial with two 8x8 grids on the screen -- a target grid, with
10 cells colored in, and a blank grid. They were told to make the blank
grid match the target grid exactly, and were unable to progress until
the grids were identical. Following this trial, participants were
informed that they would see a target grid appear on the screen for 10
seconds, followed by a picture (a visual mask) displayed for 3 seconds.
After the visual mask, participants viewed a blank 8x8 grid where they
were given 60 seconds to re-create the target grid. Participants could
click on any cell in the grid to change its color, and could also undo
any color placed. A counter on the screen showed how many targets had
been colored, and it varied dynamicaly with the participant's clicks.
After placing 10 targets, participants could click a button to adance to
the next trial. After completing 3 Practice trials, which were identical
for all partipants in all chains, participants were informed that the
study would begin.

Each participant then completed 6 Experiment trials. Participants in the
first generation of each chain received the same initial grids for
Experiment trials. These initial 8x8 grids were generated by randonly
selecting 10 of the 64 possible cells to be filled. Participants in
subsequent chains received as their targets the outputs produced by
their parent in the chain.

Participants' performance on the practice trials were used as an
attention check to determine whether their data would be passed to the
next participant. If the participant scored less than 75\% accuracy on
the last 2 practice trials, or if they failed to select 10 cells before
time ran out, their outputs were not transmitted to the next generation.
In Experiment 1, five participants failed to meet these criteria and
were excluded from analysis.

\subsection{Results and Analysis}\label{results-and-analysis}

Our primary measures of interest were reproduction accuracy and pattern
complexity. Reproduction accuracy served as a proxy for transmissibiltiy
-- higher reproduciton accuracies indicate that the ``language'' is
easier to learn. Reproduction accuracy was computed as the proportion of
targets (out of 10) which were placed in the same location on the target
and input grids.

Complexity served as a proxy for descriptiveness. We followed Kempe et
al. (2015) in using several measures of complexity: algorithmic
complexity, chunking, and edge length. Algorithmic complexity is
calculated using the Block Decomposition Method, a measure of
Kolmogorov-Chaitin Complexity applied to 2-dimensional patterns (Zenil,
Soler-Toscano, Dingle, \& Louis, 2014). This measure computes the length
of the shortest Turing machine program required to produce the observed
pattern. The shorter the program, the simpler the pattern. Chunking is
the number of groups of colored blocks which share an edge. The more
groups of blocks, the easier the pattern is to transmit, and the lower
the complexity is. Edge length is the total perimeter of the colored
blocks. If all blocks were in one chunk, the edge length would be low,
and the complexity of the pattern would likely be lower compared to if
none of the chosen targets shared an edge. Implementation of these
metrics was adapted from code provided by Gauvrit, Soler-Toscano, \&
Guida (2017).

\subsection{Results}\label{results}

If iterated learning captures the hypothesized pressures of
expressiveness and transmissibility, we predict that over generations
reproduction accuracy should increase and complexity should decrease. We
tested these predictions with mixed-effects logistic regressions,
predicting accuracy and all three measures of complexity separately from
fixed effects of generation and trial number, and random intercepts for
participant, and initial grid (e.g.
\texttt{accuracy $\sim$ generation + trial +  (1|subject) + (1|initialGrid)}.

Reproduction accuracy increased significantly over generations
(\(\beta =\) 0.033, \(t =\) 3.146, \(p =\) .002). Figure
\ref{fig:baseline_bothAccuracy} shows the results for accuracy.
Complexity on all three measure decreased significantly over generations
(\(\beta_{BDM} =\) 0.033, \(t =\) 3.146, \(p =\) .002;
\(\beta_{chunking} =\) 0.033, \(t =\) 3.146, \(p =\) .002;
\(\beta_{edge} =\) 0.033, \(t =\) 3.146, \(p =\) .002). Trial Number was
not a significant predictor in any model. Figure
\ref{fig:baseline_bothExp_withplots} shows the results for algorithmic
complexity.

\begin{CodeChunk}
\begin{figure}[tb]

{\centering \includegraphics{figs/e1_acc_plot-1} 

}

\caption[Experiments 1 and 2 show increases in accuracy over transmission generations.CHANGE POINT SIZE]{Experiments 1 and 2 show increases in accuracy over transmission generations.CHANGE POINT SIZE}\label{fig:e1_acc_plot}
\end{figure}
\end{CodeChunk}

\begin{CodeChunk}
\begin{figure}[tb]

{\centering \includegraphics{figs/e1_bdm_plot-1} 

}

\caption[Experiments 1 and 2 show increases in accuracy over transmission generations.CHANGE POINT SIZE]{Experiments 1 and 2 show increases in accuracy over transmission generations.CHANGE POINT SIZE}\label{fig:e1_bdm_plot}
\end{figure}
\end{CodeChunk}

\section{Experiment 2: Replication and extension of Experiment
1}\label{experiment-2-replication-and-extension-of-experiment-1}

Experiment 2, replicated the task from Experiment 1 with the addition of
twice as many chains and generations. We replicated the task with a
larger sample in order to approximate the shape of the algorithmic
complexity curve. Particularly, we were interested in whether complexity
asymptoted over generations.

\subsection{Method}\label{method-1}

\subsection{Participants}\label{participants-1}

Participants in Experiment 2 were 519 adults recruited on Amazon
Mechanical Turk. These participants were dived into forty diffusion
chains, each of which had twelve generations. Each participant gave
informed consent, and was compensated with \$0.50 for their
participation.

\subsection{Design and Procedure}\label{design-and-procedure-1}

The task in Experiment 2 was identical to Experiment 1. Participants
were told to reproduce patterns on a grid, and their responses were
passed to the next subject in the transmission chain.

Approximately 8\% (n=39) of participants in Experiment 2 were excluded
from analysis due to failure to meet accuracy requirements on the
practice trials or failure to select the complete number of targets on
one or more experimental trials. This resulted in a total of 480
participants included in the analysis.

\subsection{Results}\label{results-1}

The results of this experiment replicated those found in Experiment 1.
Reproduction accuracy increased significantly over generations
(\(\beta =\) 0.061, \(t =\) 6.07, \(p =\) \textless{} .001). Figure
\ref{fig:baseline_bothAccuracy} shows the results for accuracy.

Figure \ref{fig:baseline_bothExp_withplots} shows the results for
algorithmic complexity. Algorithmic complexity appeared to follow
(W.C.??) an exponential function of the form \(y = e^{-x} + b\). We
therefore fit an exponential mixed-effects regression model predicting
complexity from fixed effects of generation and trial number, and random
intercepts for participant, chain, and initial grid (e.g.
\texttt{log(complexity) $\sim$ generation + trial + (1|subject) + (1|initial) + (1|chain)}).
Algorithmic complexity decreased and asymptoted over generations
(\(\beta_{BDM} =\) -0.046, \(t =\) -12.368, \(p =\) \textless{} .001).
Similar trends were also found with chunking and edge length, the
alternate measures of complexity (CHECK THESE MODELS B/C THE LOG
EXPONENTIAL DOESN'T WORK \(\beta_{chunking} =\) -0.881, \(t =\) -17.286,
\(p =\) \textless{} .001; \(\beta_{edge} =\) -1.76, \(t =\) -15.766,
\(p =\) \textless{} .001).

\begin{CodeChunk}
\begin{figure}[tb]

{\centering \includegraphics{figs/e2_acc_plot-1} 

}

\caption[Experiments 1 and 2 show increases in accuracy over transmission generations.CHANGE POINT SIZE]{Experiments 1 and 2 show increases in accuracy over transmission generations.CHANGE POINT SIZE}\label{fig:e2_acc_plot}
\end{figure}
\end{CodeChunk}

\begin{CodeChunk}
\begin{figure}[tb]

{\centering \includegraphics{figs/e2_bdm_plot-1} 

}

\caption[Experiments 1 and 2 show increases in accuracy over transmission generations.CHANGE POINT SIZE]{Experiments 1 and 2 show increases in accuracy over transmission generations.CHANGE POINT SIZE}\label{fig:e2_bdm_plot}
\end{figure}
\end{CodeChunk}

\section{Experiment 3: Introducing an
interlocutor}\label{experiment-3-introducing-an-interlocutor}

In order to add an element of feedback from a more experienced
interlocutor to the iterated-learning process, we adapted the task from
Experiments 1 and 2 to include a secondary, ``editing'' participant.
This participant was analogus to a caregiver who protects their child
from learning incorrect forms of language.

\subsection{Method}\label{method-2}

\subsection{Participants}\label{participants-2}

Participants in Experiment 3 were 1031 adults recruited on Amazon
Mechanical Turk. These participants were dived into forty diffusion
chains, each of which had twelve generations. Each participant gave
informed consent, and was compensated with \$0.50 for their
participation.

\subsection{Design and Procedure}\label{design-and-procedure-2}

In the third, dyad experiment, a primary participant was designated to
be a ``learner'', and completed the same task as in Experiment 1 and
Experiment 2. They were told to re-produce patterns on a grid. A
secondary participant -- the ``fixer'' -- was given an adapted task.
Throughout the study, fixers were not told to re-create patterns, but to
fix patterns to resemble a target grid exactly. Fixers in this
experiment viewed the same target grid as learners, but instead of
seeing an empty input grid, they saw a grid with 10 elements filled in
-- the elements that the previous learner had submitted. The participant
could then edit the 10 items' positions. There was no ``reset'' button
during this task, so produced patterns reflect participants' initial
instincts.

In Experiment 3, a generation consisted of a learner, who re-created the
target grid, and a fixer, who then received the same target grid as well
as the learner's input grid to edit. The fixer's edited pattern was used
as the target grid for the subsequent generation.

Approximately 8\% (n=71 of participants in Experiment 3 were excluded
from analysis due to failure to meet accuracy requirements on the
practice trials or failure to select the necessary number of targets on
one or more experimental trials. This resulted in a total of 960
participants included in the analysis.

\subsection{Analysis and Results}\label{analysis-and-results}

As in Experiments 1 and 2, our primary measures of analysis were
accuracy and complexity. These measures were computed using the same
methods as in the previous experiments.

Fixers and learners had significantly different pattern reproduction
accuracies \ref{fig:dyad_accuracy}. According to a linear mixed-effects
model (need to put in formula? Did an lmer with gen and condition
(learner/fixer) as fixed effects, with all the same random effects), the
accuracies between groups were sigificantly different
(\(\beta_{condition-child} =\) -0.087, \(t =\) -11.38, \(p =\)
\textless{} .001). The fixers' transmission accuracies did not increase
significantly over generations (\(\beta_{fixers} =\) 0.01, \(t =\)
1.135, \(p =\) .257), while the accuracy of the learners showed a
marginally significant increase (\(\beta_{learners} =\) 0.014, \(t =\)
1.511, \(p =\) .132).

\ref{fig:dyad_complexity} shows the relationship between the complexity
of fixers' and learners' patterns. In each generation, the learner
decreases the complexity of the pattern, and the fixer is able to
compensate for some of this loss. AS in Experiment 2, we fit an
exponential model to the data. Both conditions show decreases in pattern
complexity over generations (\(\beta_{learners} =\) -0.029, \(t =\)
-4.498, \(p =\) \textless{} .001; \(\beta_{fixers} =\) -0.018, \(t =\)
-6.831, \(p =\) \textless{} .001), although the effect of generation is
stronger for learners compared to fixers (\(\beta_{generation} =\)
-4.843, \(t =\) -9.869, \(p =\) \textless{} .001). These results hold
true for all three measures of complexity (Do i need to report all of
these stats??).

\ref{fig:both_complexity} shows that the presence of an editor does help
retain complexity in the grid patterns. The addition of a fixer into the
task allowed a higher degree of complexity to be retained in the
language over time (\(\beta_{condition-child} =\) -3.584, \(t =\)
-6.155, \(p =\) \textless{} .001). Additionally, it appeared that the
patterns in the dyad condition asymptoted sooner than in the baseline
condition (Stats for this?).

\begin{CodeChunk}
\begin{figure}[tb]

{\centering \includegraphics{figs/dyad_accuracy-1} 

}

\caption[In the dyad task, reproduction accuracy stays relatively constant across generations]{In the dyad task, reproduction accuracy stays relatively constant across generations. Fixers have significantly higher accuracies than learners. CHANGE POINT SIZE}\label{fig:dyad_accuracy}
\end{figure}
\end{CodeChunk}

\begin{CodeChunk}
\begin{figure}[tb]

{\centering \includegraphics{figs/dyad_complexity-1} 

}

\caption[Fixers reintroudce algorithmic complexity which is lost by learners in the dyad condition]{Fixers reintroudce algorithmic complexity which is lost by learners in the dyad condition.}\label{fig:dyad_complexity}
\end{figure}
\end{CodeChunk}

\begin{CodeChunk}
\begin{figure}[tb]

{\centering \includegraphics{figs/both_complexity-1} 

}

\caption[The presence of a fixer in the dyad condition causes a much greater level of algorithmic complexity to be retained across the evolution of a novel language]{The presence of a fixer in the dyad condition causes a much greater level of algorithmic complexity to be retained across the evolution of a novel language. CHANGE POSITION IN PAPER}\label{fig:both_complexity}
\end{figure}
\end{CodeChunk}

\begin{CodeChunk}
\begin{figure}[tb]

{\centering \includegraphics{figs/e3_acc_plot-1} 

}

\caption[Experiments 1 and 2 show increases in accuracy over transmission generations.CHANGE POINT SIZE]{Experiments 1 and 2 show increases in accuracy over transmission generations.CHANGE POINT SIZE}\label{fig:e3_acc_plot}
\end{figure}
\end{CodeChunk}

\begin{CodeChunk}
\begin{figure}[tb]

{\centering \includegraphics{figs/e3_bdm_plot-1} 

}

\caption[Experiments 1 and 2 show increases in accuracy over transmission generations.CHANGE POINT SIZE]{Experiments 1 and 2 show increases in accuracy over transmission generations.CHANGE POINT SIZE}\label{fig:e3_bdm_plot}
\end{figure}
\end{CodeChunk}

\section{General Discussion}\label{general-discussion}

Although Experiments 1-3 used a non-linguistic task, we were able to
measure change in a culturally-transmitted, learned symbol system. In
Experiments 1 and 2, language simplified rapidly and dramatically,
reflecting the strong pressure towards simplification in language
learning. These findings replicated those of Kempe et al. (2015): when
transmitting an artifical language of grid patterns, complexity in the
language was lost.

However, the results of Experiment 3 show that this loss is not
permanent, but can be reintroduced in the language by way of a secondary
participant who helps bring the language towards a stable level of
complexity. When the iterated-learning process begins to resemble the
true process of language-learning, where children speak with and are
subject to correction by those more competent in the language, a lesser
amount of complexity was lost during transmission. Additionally, this
stable level of complexity is much higher, and is reached earlier in the
transmission chain with the help of a fixing participant. This stability
in complexity did not mean that the language stopped changing, but that
the descriptiveness and transmissibility pressures were in balance.
Fixers in Experiment 3 represented caregivers -- they were more accurate
at reproducing the language, and could therefore be seen as more fluent
speakers of the language, just as adults are of their native languages.
The learners, on the other hand, had a more difficult task, which
greater strained their working memories, similar to the strain on a
child language learner who is inundated with new words each day. The
fixer's corrected language was passed to the next learner in the chain,
representing a child who, after many years of being corrected by their
own parent, becomes a parent, and, in turn, passes their optimal
language to the next generation. Due to the higher accuracy by fixers,
and therefore greater knowledge of the language, the fixers were were
able to compensate for some (not all) of the loss in complexity seen by
the learners by editing their patterns.

In Experiment 3, the learner's reproduction accuracies were actually
increasing over generations. Despite the stable level of complexity,
learners found the language easier to reproduce over evolution. Although
a high level of descriptiveness was retained in the language,
transmissibility was increasing, without the simplicity pressure
weighing in. Perhaps the language was becoming stable and complex, with
the symbol-patterns changing to be both descriptive and useful, while
being easily transmissible. This reflects the optimal evolutionary
response to these two competing pressures.

When a caregiver or teacher prevents their child from growing up to
believe that ``baba'' is the word for both ``bottle'' and ``sheep'',
they are not only helping their individual child become a competent
speaker of the language, but they are also re-introducing complexity,
and helping the language system as a whole from simplifying to disuse.
Data collection is ongoing with children ages 6-8 at a local science
museum in both the Experiment 2 and Experiment 3 tasks, in order to
investigate whether the pressures of similarity and complexity affect
children similarly to how they affect adults in early language-learning
conditions.

We do not learn language as passive listeners, who absorb a proportion
of the the linguistic input they hear. Therefore, we cannot measure
language learning only through measuring input, nor through measuring
only linguistic output. Languages are both learned and changed through
conversations, with feedback and error correction, to evolve to the
needs of the language's users. Therefore, we must study language
learning in process, to see how it adapts and evolves with communicative
interactions.

\vspace{1em}
\fbox{\parbox[b][][c]{7.3cm}{\centering All code for these analyses are available at\ \url{https://github.com/mcmeyers/iteratedlearning}}}

\section{Acknowledgements}\label{acknowledgements}

This research was funded by a James S. McDonnell Foundation Scholar
Award to DY.

\section{References}\label{references}

\setlength{\parindent}{-0.1in} \setlength{\leftskip}{0.125in}

\noindent

\hypertarget{refs}{}
\hypertarget{ref-bowerman-1982}{}
Bowerman, M. (1982). U shaped behavioral growth. In (pp. 101--145).
Academic Press.

\hypertarget{ref-chouinard-2003}{}
Chouinard, M. M., \& Clark, E. V. (2003). Adult reformulations of child
errors as negative evidence. \emph{Journal of Child Language},
\emph{30}(3), 637--669.

\hypertarget{ref-christiansen-2003}{}
Christiansen, M. H., \& Kirby, S. (2003). Language evolution. In M. H.
Christiansen \& S. Kirby (Eds.), (pp. 1--15). Oxford University Press.

\hypertarget{ref-gauvrit-2017}{}
Gauvrit, N., Soler-Toscano, F., \& Guida, A. (2017). A preference for
some types of complexity comment on ``perceived beauty of random texture
patterns: A preference for complexity''. \emph{Acta Psychologica},
\emph{174}, 48--53.

\hypertarget{ref-hudsonkam-2005}{}
Hudson Kam, C. L., \& Newport, E. L. (2005). Regularizing unpredictable
variation: The roles of adult and child learners in languagae formation
and change. \emph{Language Learning and Development}, \emph{1}(2),
151--195.

\hypertarget{ref-kempe-2015}{}
Kempe, V., Gauvrit, N., \& Forsyth, D. (2015). Structure emerges faster
during cultural transmission in children than in adults.
\emph{Cognition}, \emph{136}, 247--254.

\hypertarget{ref-kirby-2007}{}
Kirby, S., Dowman, M., \& Griffiths, T. L. (2007). Innateness and
culture in the evolution of language. \emph{Proceedings of the National
Academy of Sciences}, \emph{104}(12), 5241--5245.

\hypertarget{ref-kirby-2014}{}
Kirby, S., Griffiths, T., \& Smith, K. (2014). Iterated learning and the
evolution of language. \emph{Current Opinion in Neurobiology},
\emph{28}, 108--114.

\hypertarget{ref-lupyan-2010}{}
Lupyan, G., \& Dale, R. (2010). Language structure is partly determined
by social structure. \emph{PLoS ONE}, \emph{5}(1), 1--10.

\hypertarget{ref-penner-1987}{}
Penner, S. G. (1987). Parental responses to grammatical and
ungrammatical child utterances. \emph{Child Development}, \emph{58}(2),
376--384.

\hypertarget{ref-raviv-2018}{}
Raviv, L., \& Arnon, I. (2018). Systematicity, but not compositionality:
Examining the emergence of linguistic structure in children and adults
using iterated learning. \emph{Cognition}, \emph{181}, 160--173.

\hypertarget{ref-regier2015}{}
Regier, T., Kemp, C., \& Kay, P. (2015). 11 word meanings across
languages support efficient communication. \emph{The Handbook of
Language Emergence}, \emph{87}, 237.

\hypertarget{ref-romberg-2010}{}
Romberg, A. R., \& Saffran, J. (2010). Statistical learning and language
acquisition. \emph{WIREs Cognitive Science}, \emph{1}, 906--914.

\hypertarget{ref-senghas-2003}{}
Senghas, A. (2003). Intergenerational influence and ontogenetic
development in the emergence of spatial grammar in nicaraguan sign
language. \emph{Cognitive Development}, \emph{18}, 511--531.

\hypertarget{ref-smith-2010}{}
Smith, K., \& Wonnacott, E. (2010). Eliminating unpredictable variation
through iterated learning. \emph{Cognition}, \emph{116}, 444--449.

\hypertarget{ref-verhoef-2014}{}
structure, E. of combinatorial, \& signals. (2014). Verhoef, tessa and
kirby, simon and de boer, bart. \emph{Journal of Phonetics}, \emph{43},
57--68.

\hypertarget{ref-zenil-2014}{}
Zenil, H., Soler-Toscano, F., Dingle, K., \& Louis, A. A. (2014).
Correlation of automorphism group size and topolical properties with
program-size complexity evaluations of graphs and complex networks.
\emph{Physica A}, \emph{404}, 341--358.

\bibliographystyle{apacite}


\end{document}
